\documentclass[tikz]{standalone}
\usepackage{pgfplots}
\begin{document}
	\usetikzlibrary{decorations.pathmorphing}
	\usetikzlibrary{shapes.geometric,calc,positioning}
	\usetikzlibrary{arrows.meta}
	\usetikzlibrary{shapes,snakes}
	\usetikzlibrary{decorations.markings}
	
	


	
	\begin{tikzpicture}
	
	\draw[step=1,gray,very thin] (-10,-10) grid (10,10);
		\def\dotWidth{50}
		\def\electronSize{20}
		\def\spinLength{50}
		\def\leadHeight{100}
		\def\leadWidth{5}
		\def\fermiWidth{50}
		
		\tikzstyle{dot}[black] = [draw, align=center,
          ultra thick, fill=#1, color=#1]
\tikzstyle{electron}[red] = [shading=ball, ball color=#1, 
                     circle, minimum size=\electronSize]
\tikzstyle{hole}[red] = [electron=#1, ultra thick, dashed, draw=#1, fill
opacity=0.5  ]
\tikzstyle{spinUpArrow} = [fill=red, single arrow,
                   minimum height=\spinLength, minimum 
                   width=2/5*\spinLength,
                   single arrow head extend=2/50*\spinLength,
                   align=center, rotate=90]
\tikzstyle{spinDownArrow} = [fill=blue, single arrow,
     minimum height=\spinLength, minimum 
     width=20mm,
     single arrow head extend=2mm,
     align=center, rotate=270]
\tikzstyle{control} = [align=center,draw, thick, blue, inner 
sep=0]



\tikzset{
	spin up/.style ={fill=#1, minimum height=50, minimum width=20, single 
	arrow, 
					 single arrow head extend=2,
				     align=center, rotate=90},
	spin down/.style={fill=#1, minimum height=50, minimum width=20, 
	single arrow, 
	                  single arrow head extend=2,
				      align=center, rotate=270},
	lead barrier/.style={thin, draw=black, fill=black!20, minimum 
						 height=\leadHeight, minimum width=\leadWidth},
	lead/.style={thin,color=black,fill=black!10}
}

%\newcommand{\fermi}{
%	\begin{tikzpicture}[xscale=1.0,yscale=1.0]
%		\draw[domain=-1.0:1.0,smooth,variable=\x, lead] plot 
%		({1-1/(exp(\x*7.2+2.5)+1)},\x) 
%		{ -- ++(0,-2)}
%		;
%	\end{tikzpicture}%
%}
%\newcommand\gauss[2]{1/(#2*sqrt(2*pi))*exp(-((x-#1)^2)/(2*#2^2))}
\newcommand\fermidirac[2]{1/(exp((#1*x-#2))+1)}
\newcommand{\fermi}{
		\begin{axis}[every axis plot post/.append style={
		 domain=-3:1,samples=50,smooth}, 
		   hide axis,
		   no markers,
		   clip bounding box=upper bound,
%		   axis line style={draw=none}, tick style={draw=none},
%		xticklabels={,,},
%		yticklabels={,,},		
		axis x line*=bottom, % no box around the plot, only x and y axis
		axis y line*=left % the * suppresses the arrow tips
	] % extend the axes a bit to the right and top
		\addplot[thick, color=black, fill=black!10] {\fermidirac{10}{1}} 
		\closedcycle;
		\end{axis}
	%	\draw[domain=-1.0:1.0,smooth,variable=\x, lead] plot 
%		({1-1/(exp(\x*7.2+2.5)+1)},\x) 
}


		
		\node[control] at (-2,2) (up) {};
		\node[control] at (2, -2) (down) {};
		\node[control] at (-6, 0) (leadL) {};	
		\node[control] at (6, 0) (leadR) {};
		\node[control] at (-8, 2) (holeL) {};
		\node[control] at (8, -2) (holeR) {};
		\node[control] at (-8,4) (fermiL) {};
		\node[control] at (8,-4) (fermiR) {};			
				
		\node[dot] at (up) {};
		\node[dot] at (down) {};
	
		\node[lead barrier] (leadL) at (leadL) {};
		\node[lead barrier] at (leadR) {};	
				
		\node[inner sep=0, anchor=south east, rotate=90] (fermiL) at 
		(leadL.north west) 
		{
			\begin{tikzpicture}[inner sep=0]
			\fermi
			\end{tikzpicture}
			};
		
		\node[spin up=red] at (up) {};
		\node[spin down=blue] at (down) {};

		\node[electron=red] at (up) {};
		\node[electron=blue] at (down) {};	
			
	
		\node[hole=red] at (holeL) {};
		\node[hole=blue] at (holeR) {};
		

		
	\end{tikzpicture}
\end{document}
